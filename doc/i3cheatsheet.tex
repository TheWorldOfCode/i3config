\documentclass{article}

\usepackage{listings}

\title{i3 Cheat sheet}

\begin{document}
\maketitle
The super key is the key with the Windows flag. 

\section*{Basic Movement}
\begin{itemize}
	\item \verb|Mod+h/j/k/l| -- Move to different windows (Vim style)
	\item \verb|Mod+H/J/K/L| -- Move the focus window around (Vim style)
        \item \verb|Mod+{number}| -- Change workspace to number
	\item \verb|Mod+Shift+{number}| -- Move the focused window to workspace \{number\}
	\item \verb|Mod+space| -- Change focus between tiling and floating windows
	\item \verb|Mod+a| -- Focused the parent container 
	\item \verb|Mod+A| -- Focused the child container
	\item \verb|Mod+Shift+minus| -- Move the focused window to scratchpad 
	\item \verb|Mod+minus| -- Show scratchpad and cycles through the window in it.
	\item \verb|Mod+f| -- Show the focused window in fullscreen mod
	\item \verb|Mod+r| -- Enter resize mode, resize with \verb|h/j/k/l|, use \verb|Return| or \verb|Escape|a return to normal mode
\end{itemize}

\section*{Layout}

\begin{itemize}
	\item \verb|Mod+b| -- Split horizontal 
	\item \verb|Mod+v| -- Split vertical 
	\item \verb|Mod+s| -- Change container layout to stacking
	\item \verb|Mod+w| -- Change container layout to tabbed
	\item \verb|Mod+e| -- Toggle split on container (Vertical/Horizontal)
	\item \verb|Mod+Shift+space| -- Toggle tiling and floating
\end{itemize}

\section*{Programs}
\begin{itemize}
	\item \verb|Mod+Return| -- Start x-terminal-emulator (Read section \ref{sec:alternatives})
	\item \verb|Mod+W| -- Start x-www-browser (Read section \ref{sec:alternatives}))
	\item \verb|Mod+x| -- Start vifm (file browser)
	\item \verb|Mod+X| -- Start htop (System info)
	\item \verb|Mod+f2| -- Start mutt (Mail)
	\item \verb|Mod+d| -- Start dmenu (program launcher)
\end{itemize}

\section*{System}
\begin{itemize}
	\item \verb|Mod+f1| -- Open this document (the Cheat sheet)
	\item \verb|Mod+Shift+c| -- Reload the configuration file
	\item \verb|Mod+Shift+r| -- Restart i3 in place (??)
	\item \verb|Mod+Shift+e| -- Log out
	\item \verb|Mod+Shift+q| -- Kill the focused window
	\item \verb|Mod+Ctrl+l| -- Lock the screen
\end{itemize}

The function keys that works are 
\begin{itemize}
	\item \verb|XF86AudioRaiseVolume| -- Increase Master volume by 5 \% and unmute
	\item \verb|XF86AudioLowerVolume| -- Decrease Master volume by 5 \% and unmute
	\item \verb|XF86AudioMute| -- Toggle the Master volume
	\item \verb|XF86AudioPlay| -- Start or pause the media player 
	\item \verb|XF86AudioNext| -- Next track  
	\item \verb|XF86AudioPrev| -- Previous track  
\end{itemize}

\section{Program alternatives}
\label{sec:alternatives}
Many of the shortcuts uses the Debian alternatives system, which allow to easy define default program to different action, such as browser and terminal.

\begin{lstlisting}[languagae=bash] 
  update-alternatives --list [link_group] ] 
\end{lstlisting}

To change program used following command
\begin{listlisting}[languagae=bash]
   sudo update-alternatives --config [link_grop] 
\end{listlisting}

This command will show a list for alternatives where each has a selection number, enter the selection number of the want program to be used. \\

To create or add programs to a link group run the following command

\begin{listlisting}[languagae=bash]
   sudo update-alternatives --install [link_path]  [link_group] [program_path] [priority]  
\end{listlisting}
where \verb|[link_path]| is the location where the link should be placed, \verb|link_group| the name of the link group the program should be added to, \verb|[program_path]| the path to the program, \verb|[priority]| the priority in the group. 

Used alternatives, group that are marked is manual created.
\begin{itemize}
	\item \verb|x-www-browser|: Change the default web browser
	\item \verb|x-terminal-emulator|: Change the default terminal emulator
	\item \verb|x-pdfviewer|: The default pdf viewer (created)
\end{itemize}

\end{document}

